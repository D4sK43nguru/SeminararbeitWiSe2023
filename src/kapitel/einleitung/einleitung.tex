\section{Einleitung}\label{sec:einleitung}

35 Jahre nach dem ersten Cyberangriff und 24 Jahre nach dem ersten \ac{ddos}"~Angriff werden regelmäßig neue Sicherheitslücken in Netzwerken und Programmen gefunden und ausgenutzt.
Die Anzahl der Mal- und Ransomware Angriffe sank zwar durch die Coronapandemie ein wenig, war davor jedoch auf einem historischen Maximum.
Zudem ist der Anteil der Unternehmen, die von einem Cyberangriff betroffen waren, so hoch wie noch nie.
Auch die schnelle Verbreitung von \ac{iot}"~Geräten erfordert immer stärker das Absichern von Netzwerken, um sowohl die Infrastruktur, als auch den Geräten zugängliche Daten zu schützen.
So sind viele Unternehmen, besonders solche die auf kritischer Infrastruktur basieren oder mit wichtigen Daten arbeiten, an einem möglichst idealen Schutz gegenüber diese Angriffe interessiert.

\subsection{Zielsetzung}\label{subsec:zielsetzung}
Diese Seminararbeit untersucht die Auswirkungen einer \ac{zta} auf die Bekämpfung moderner Cyberbedrohungen.
Dabei wird untersucht, wie die Implementierung einer \ac{zta} die Häufigkeit von Cyberangriffen und von diesen verursachten Datenverlusten in Unternehmen beeinflusst.
Zudem werden die möglichen Auswirkungen auf die Benutzerfreundlichkeit und Produktivität von Mitarbeitern betrachte.

Die erwarteten Ergebnisse umfassen eine potenzielle Verringerung von Cyberangriffen und Datenverlusten, da eine \ac{zta} eine strikte Überprüfung von Netzwerkzugriffen bietet.
Zugleich werden mögliche Einflüsse auf die Benutzerfreundlichkeit und Produktivität der Mitarbeiter untersucht, um einen ausgewogenen Ansatz zwischen sicherheit und Arbeitsleistung zu finden.

Zur Erstellung der Seminararbeit wird primär Literaturarbeit durchgeführt, welche sich auf eine systematische Analyse und Synthese bestehender wissenschaftlicher Quellen und Publikationen stützt.
Hierzu wird zunächst ausführlich nach bestehender Literatur recherchiert, welche dann nach Relevanz für das Thema selektiert wird.
Anschließend werden die gewählten Quellen sorgfältig gelesen und analysiert.
Die relevanten Informationen dieser werden extrahiert, dies beinhaltet Daten, Fallstudien und Expertenmeinungen.

Die gesammelten Ergebnisse werden darauf in einem ganzheitlichen Ansatz zusammengeführt, um die Forschungsfragen zu beantworten und die zu erwartenden Ergebnisse zu entwickeln.
Zuletzt werden die Stärken und Schwächen der identifizierten Literatur kritisch bewertet, um die Glaubwürdigkeit und Relevanz der verwendeten Quellen sicherzustellen.

Die Wahl der methodischen Herangehensweise ermöglicht eine gründliche Untersuchung des Themas, indem sie auf etablierte wissenschaftliche Erkenntnisse und Fachwissen zurückgreift.
Dies gewährleistet eine fundierte und objektive Analyse der Auswirkungen einer Zero"~Trust Netzwerkarchitektur auf moderne Cyberbedrohungen und die Mitarbeitererfahrung.

\subsection{Aufbau der Arbeit}\label{subsec:aufbau-der-arbeit}
\autoref{sec:grundlagen} führt verschiedene Grundlagen für die Ausarbeitung dieser Seminararbeit ein.
Zunächst werden \acp{zta} definiert, sowie die Entwicklung und der historische Hintergrund dieser erklärt.
Darauf werden Prinzipien von \ac{zta} dargestellt und ein Vergleich zu anderen Sicherheitsansätzen gebildet.

\autoref{sec:moderne-cyberbedrohungen} stellt verschiedene Arten von Cyberbedrohungen wie Malware- oder Phishingangriffe dar.
Zudem werden die Trends in der Cyberkriminalität erläutert.

In \autoref{sec:zta-im-detail} wird der Aufbau und die Komponenten von \ac{zta}, sowie die Implementierung dieser in Unternehmen gezeigt.
Außerdem werden Vor- und Nachteile von Zero-Trust gegeneinander aufgewogen.

Zuletzt werden in \autoref{sec:auswirkungen-und-resultate-von-zero-trust} die Auswirkungen auf die Sicherheit von Netzwerke, sowie die Reduzierung der Angriffsfläche erläutert.
Zudem belichtet dieser Abschnitt, wie Zero-Trust sensible Daten schützt, zeigt die Messbarkeit der Auswirkungen auf die Sicherheit und stellt Probleme dar.
\newpage


\section{Grundlagen}\label{sec:grundlagen}
\subsection[Definition einer Zero-Trust-Architektur]{Definition einer \ac{zta}}\label{subsec:definition-einer-zta}
\ac{zt} bezeichnet eine Sammlung an Maßnahmen der Cybersicherheit, welche darauf basieren, die Verteidigungen von netzwerkbasierten Umfängen auf Nutzer und Ressourcen umzuleiten\autocite[\vglf][\pagef 4]{NIST:800207}.
Darüber hinaus ist eine \ac{zta} ein Cybersicherheitsplan einer Einrichtung, der die Konzepte von \ac{zt} umsetzt und Zugriffsrichtlinien, Arbeitsabläufe und Beziehungen zwischen Komponenten umfasst.\autocite[\vglf][\pagef 4]{NIST:800207}

Das Ziel einer \ac{zta} ist es, \glsdisp{autorisierung}{unautorisierten} Zugriff auf Daten und Leistungen zu verhindern und hierbei das Durchführen der Zugriffskontrollen so detailliert wie möglich zu gestalten.


\subsection{Entwicklung und historischer Hintergrund}\label{subsec:entwicklung-und-historischer-hintergrund}
Zscaler hat im Jahr 2022 die Entwicklung des \ac{zt}-Konzepts knapp zusammengefasst.\autocites[\vglf][]{zscaler-2022b}
Die ersten Grundsteine, die zur Entwicklung des \ac{zt}-Konzepts führten, wurden $1987$ gelegt.
In diesem Jahr wurde von Entwicklern der \ac{dec} eine Studie zum Thema Firewall-Technologie veröffentlicht, wodurch die \enquote{Festung mit Burggraben} als Standardmodell der Netzwerksicherheit etabliert wurde.\autocites[\vglf][]{zscaler-2022b}

Im Jahr $1994$ wurde der Begriff \enquote{\ac{zt}} im Rahmen einer Doktorarbeit über Computersicherheit geprägt.
Der Autor untersuchte Vertrauen als ein mathematisch beschreibbares, endliches Gut.\autocite[\vglf][]{marsh-1994}
Im selben Jahr wurden Firewalls als eine harte Schale um einen weichen Kern, wie bei einem Ei, beschrieben.
Wenn ein Angreifer die Firewall überwinden kann, stünde ihm das ganze Netzwerk zur Verfügung. \autocite[\vglf][\pagef 29]{world-1994}

Die erste Version des 802.1X-Protokolls\autocite[\vglf][]{IEEE-2001} wurde im Jahr $2001$ veröffentlicht und als Standard für Netzzugangskontrollen eingeführt.
In diesem Protokoll wird die Authentifizierung von Netzverbindungen sowie die Vergabe von Zugriffsrechten auf Netzwerkebene vorgesehen.
Dadurch, dass das Protokoll komplexe Vorgänge vorschreibt, war es nicht zur allgemeinen Implementierung geeignet.\autocite[\vglf][]{zscaler-2022b}
Zusätzlich wurde auch in $2001$ die erste Version des \ac{osstmm} mit dem Fokus auf Vertrauen veröffentlicht.\autocite[\vglf][]{osstmm-2001}

Um das Jahr $2007$ wurde die dritte Version des \ac{osstmm} veröffentlicht, welche in einem ganzen Kapitel über die Schwachsetlle \enquote{Vertrauen} in Systemen befasst.\autocite[\vglf][]{osstmm-2010}

Eine weitere Prägung des Begriffes \enquote{\ac{zt}} fand im Jahr $2010$ statt, als der Analyst Kindervag in einem Forschungsbeitrag die Verlagerung der Authentifizierung und Cybersicherheit in den Datenpfad vorsieht und auch die Segmentierung zwischen einzelnen Sitzungen fordert.
Weiterhin wird das Paradigma des Netzwerkzugangs verhaftet, der Sicherheitsperimeter wird allerdings ins Netzwerk verschoben.\autocites[\vglf][]{zscaler-2022b}[\vglf][]{kindervag-2010}

$2018$ arbeiteten Forscher von \ac{nist} und \ac{nccoe} zusammen an einem Projekt, welches zur Veröffentlichung der \ac{nist} Leitlinie SP 800--207\autocite{NIST:800207} im Jahr $2020$ als ein erstes einheitliches Framework für \acp{zta} führte.
Diese definiert \ac{zt} als eine Sammlung verschiedener Konzepte und Ideen, welche die Unsicherheit bei anfragenbezogenen Zugriffsentscheidungen in Informationssystemen verringern sollen.
Diese Veröffentlichung leitete einen Paradigmenwechsel ein, da erstmals \ac{zt} nicht mehr im Kontext des Netzwerkzugangs definiert wird.\autocite[\vglf][]{zscaler-2022b}

Im Jahr $2021$ hat das \ac{ncsc} im Vereinigten Königreich empfohlen, dass Netzwerkarchitekten einen \ac{zt} Ansatz für neue IT-Lösungen in Anbetracht nehmen, dies besonders dann, wenn signifikante Nutzung von Cloud-Services geplant ist.\autocite{ncsc-2021}
Zusätzlich wurden im Jahr $2022$ alle US-Behörden zur Umstellung auf \ac{zt} bis $2024$ verpflichtet.\autocite[\vglf][]{zscaler-2022b}


\subsection{Prinzipien und Konzepte von Zero-Trust}\label{subsec:prinzipien-und-konzepte-von-zero-trust}
In einem System, welches \ac{zt} implementiert, muss jede Anfrage bevor sie auf die Ressourcen zugreifen kann, in einem \gls{pdp}/\gls{pep} überprüft werden\autocite[\vglf][\pagef 4]{NIST:800207}.

Eine \ac{zta} wird mit dem Gedanken entwickelt und umgesetzt, die folgenden Grundsätze umzusetzen:
\begin{itemize}
    \item Alle Datenquellen und Rechenleistungen werden als Ressourcen angesehen,
    \item Unabhängig der Netzwerkposition ist jegliche Kommunikation gesichert,
    \item Der Zugriff auf einzelne Ressourcen erfolgt auf einer Pro-Sitzung Grundlage,
    \item Zugriff auf Ressourcen wird durch dynamische Regelungen festgelegt und kann von verschiedenen Attributen, wie die Identität des \glspl{client} beeinflusst werden,
    \item Das Unternehmen überwacht und misst die \gls{integritaet} aller eigenen und verbundenen Ressourcen,
    \item Jegliche Ressourcen \gls{authentifizierung} und \gls{autorisierung} ist dynamisch und erzwungen, bevor Zugriff gewährt werden kann,
    \item Das Unternehmen sammelt so viele Informationen über den aktuellen Status der Ressourcen, Netzwerkinfrastruktur und Kommunikationen wie möglich und nutzt diese, um die Sicherheit zu erhöhen.\autocite[\vglf][\pagef 6-7]{NIST:800207}
\end{itemize}

\newpage


\section{Grundlagen}\label{sec:grundlagen}
\subsection[Definition einer Zero-Trust-Architektur]{Definition einer \gls{zta}}\label{subsec:definition-einer-zta}
\gls{zt} bezeichnet eine Sammlung an Maßnahmen der Cybersicherheit, welche darauf basieren, die Verteidigungen von netzwerkbasierten Umfängen auf Nutzer und Ressourcen umzuleiten\autocite[\vglf][\pagef 4]{NIST:800207}.
Darüber hinaus ist eine \gls{zta} ein Cybersicherheitsplan einer Einrichtung, der die Konzepte von \gls{zt} umsetzt und Zugriffsrichtlinien, Arbeitsabläufe und Beziehungen zwischen Komponenten umfasst.\autocite[\vglf][\pagef 4]{NIST:800207}

Das Ziel einer \gls{zta} ist es, \glsdisp{autorisierung}{unautorisierten} Zugriff auf Daten und Leistungen zu verhindern und hierbei das Durchführen der Zugriffskontrollen so detailliert wie möglich zu gestalten.

\subsection{Entwicklung und historischer Hintergrund}\label{subsec:entwicklung-und-historischer-hintergrund}
\todo[inline]{Überlegen, ob dieses Kapitel im Text bleibt oder entfent wird}

\subsection{Prinzipien und Konzepte von Zero-Trust}\label{subsec:prinzipien-und-konzepte-von-zero-trust}

In einem System, welches \gls{zt} implementiert, muss jede Anfrage bevor sie auf die Ressourcen zugreifen kann, in einem \gls{pdp}/\gls{pep} überprüft werden\autocite[\vglf][\pagef 4]{NIST:800207}.

Eine \gls{zta} wird mit dem Gedanken entwickelt und umgesetzt, die folgenden Grundsätze umzusetzen:
\begin{itemize}
    \item Alle Datenquellen und Rechenleistungen werden als Ressourcen angesehen,
    \item Unabhängig der Netzwerkposition ist jegliche Kommunikation gesichert,
    \item Der Zugriff auf einzelne Ressourcen erfolgt auf einer Pro-Sitzung Grundlage,
    \item Zugriff auf Ressourcen wird durch dynamische Regelungen festgelegt und kann von verschiedenen Attributen, wie die Identität des \glspl{client} beeinflusst werden,
    \item Das Unternehmen überwacht und misst die \gls{integritaet} aller eigenen und verbundenen Ressourcen,
    \item Jegliche Ressourcen \gls{authentifizierung} und \gls{autorisierung} ist dynamisch und erzwungen, bevor Zugriff gewährt werden kann,
    \item Das Unternehmen sammelt so viele Informationen über den aktuellen Status der Ressourcen, Netzwerkinfrastruktur und Kommunikationen wie möglich und nutzt diese, um die Sicherheit zu erhöhen.\autocite[\vglf][\pagef 6-7]{NIST:800207}\todo[inline]{Eventuell ibidem entfernen mit „\string\makeatletter\string\blx\string@ibidreset\string\makeatother“}
\end{itemize}

\subsection{Vergleich zu herkömmlichen Sicherheitsansätzen}\label{subsec:vergleich-zu-herkommlichen-sicherheitsansatzen}

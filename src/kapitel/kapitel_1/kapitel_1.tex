\newpage


\section{Grundlagen}\label{sec:grundlagen}
\subsection[Definition einer Zero-Trust-Architektur]{Definition einer \ac{zta}}\label{subsec:definition-einer-zta}
\ac{zt} bezeichnet eine Sammlung an Maßnahmen der Cybersicherheit, welche darauf basieren, die Verteidigungen von netzwerkbasierten Umfängen auf Nutzer und Ressourcen umzuleiten\autocite[\vglf][\pagef 13]{NIST:800207}.
Darüber hinaus ist eine \ac{zta} ein Cybersicherheitsplan einer Einrichtung, der die Konzepte von \ac{zt} umsetzt und Zugriffsrichtlinien, Arbeitsabläufe und Beziehungen zwischen Komponenten umfasst.\autocite[\vglf][\pagef 13]{NIST:800207}

\subsection{Entwicklung und historischer Hintergrund}\label{subsec:entwicklung-und-historischer-hintergrund}

\subsection{Prinzipien und Konzepte von Zero-Trust}\label{subsec:prinzipien-und-konzepte-von-zero-trust}

\subsection{Vergleich zu herkömmlichen Sicherheitsansätzen}\label{subsec:vergleich-zu-herkommlichen-sicherheitsansatzen}

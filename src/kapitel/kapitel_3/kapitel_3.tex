\newpage
\section[Zero-Trust Architektur im Detail]{\ac{zta} im Detail}\label{sec:zta-im-detail}
Da in einer \ac{zta} keinem Gerät und keiner Anwendung vertraut wird, aber alle unterstützt werden, sind verschiedene Maßnahmen notwendig um die erwünschte und oft erforderte Sicherheit zu garantieren.
Dies geschieht durch regelmäßiges Überprüfen der \gls{authentizitaet} und \gls{autorisierung} der Systeme.\autocite[\vglf][\pagef 3]{dsilvia-2021}

\subsection{Komponenten und Aufbau von Zero-Trust}\label{subsec:komponenten-und-aufbau-von-zero-trust}
Jede \ac{zta} wird durch drei Eigenschaften definiert.
Diese stellen sicher, dass (\romannumeral 1) auf alle Ressourcen unabhängig von ihrer physischen oder logischen Position ein sicherer Zugriff erfolgen muss, (\romannumeral 2) strenge Zugriffskontrollmaßnahmen bestehen, und zuletzt (\romannumeral 3) jeglicher Netzwerkverkehr erfasst und aufgezeichnet wird.\autocite[\vglf][\pagef 2]{dsilvia-2021}

\subsubsection{Komponenten einer Zero-Trust Architektur}
Es gibt verschiedene Methoden, eine \ac{zta} einzurichten, von denen viele das Konzept teilen, die Kontrolle nahe an den Anwendungen und Nutzern zu halten anstatt sie in der Netzwerkinfrastruktur auszulagern.\autocite[\vglf][\pagef 4]{buck-2021}
Drei der Grundmethoden einer \ac{zta} sind \glsdisp{authentizitaet}{Anwendungsauthentifizierung}, \glsdisp{authentizitaet}{Gerätauthentifizierung} und Vertrauen, da eine \ac{zta} im Gegensatz zu anderen Sicherheitsinfrastrukturen die \gls{authentizitaet} regelmäßig überprüft, die Nutzergeräte überprüft und Widersprüchlichkeiten in Anwendungen der Nutzer überwacht und erkennt.\autocite[\vglf][\pagef 3]{dsilvia-2021}

\ac{zt} sollte nicht als einzelne Technologie gesehen werden, sondern stellt durch viele Anforderungen, Kontrollen und Prinzipien einen umfassenden Schutz dar, welcher selbst bei verschwimmender Grenze zwischen Privatem und Arbeit nicht minder wirkt.\autocite[\vglf][Was sind die Komponenten von Zero Trust]{akamai:online}

\subsubsection{Aufbau einer Zero-Trust Architektur}

Eine \ac{zta} lässt jede Anfrage eine Vertrauensevaluation durchgehen.
Eine solche Evaluation kann aus den folgenden, von Horne und Nair dargestellten, Schritten bestehen\autocite[\vglf][\pagef 3]{horne-2021}:
\begin{enumerate}
    \item Vertrauensfaktoren, darunter unter anderem
    \begin{itemize}
        \item \glsdisp{authentizitaet}{Nutzerauthentifizierung},
        \item Nutzerrolle oder -profil,
        \item \glsdisp{authentizitaet}{Gerätauthentifizierung},
        \item Geräteart und -status,
        \item IP-Adresse, \bzw Ort,
        \item Details der Zugriffsanfrage,
        \item Verhaltensdaten,
    \end{itemize}
    \item Vertrauensalgorithmus,
    \item Anforderungen und Anforderungsadministrator, mit den folgenden Richtlinien
    \begin{itemize}
        \item Vertrauensgrenzwerte,
        \item Grundsätze zur Einhaltung der Richtlinien,
        \item Richtlinien für Endgeräte,
        \item Datenschutzrichtlinien
    \end{itemize}
    \item Erlauben oder Ablehnen der Zugriffsanfrage.
\end{enumerate}
Die Überprüfung nach dem $4.$ Schritt der vorherigen Auflistung läuft dabei wie folgt ab:\autocite[\vglf][\pagef 3]{horne-2021}
\begin{enumerate}
    \item Dabei wird die Anfrage zunächst auf die einzelnen Faktoren überprüft und evaluiert.
    \item Anschließend wird aus dem Resultat dieser Evaluation ein Vertrauenswert berechnet, mit welchem die Anfrage überprüft und einzelne Richtlinien zugeschrieben werden.
    \item Zuletzt wird, sofern ein ausreichender Vertrauenswert vorhanden ist, die Anfrage unter Einhaltung der Richtlinien akzeptiert, andernfalls abgelehnt.
\end{enumerate}

\subsection{Implementierung von Zero-Trust in Unternehmen}\label{subsec:implementierung-von-zero-trust-in-unternehmen}
Eine \ac{zta} kann sowohl in einem neuen System implementiert werden, als auch in einem bereits bestehenden eingearbeitet werden.
Hierfür werden die folgenden Annahmen genommen:\autocite[\vglf][\pagef 3]{dsilvia-2021}
\begin{itemize}
    \item Das LAN innerhalb eines Netzwerkes sollte nicht implizit als vertraute Zone behandelt werden.
    \item Mit dem aktuellen Trend, dass in Unternehmen \ac{byod} eingeführt wird, wird davon ausgegangen, dass Geräte, die mit dem Netzwerk verbunden sind, keine Instanz des Unternehmens sind, da jedes Gerät manipuliert werden kann.
    \item Ressourcen sind niemals vertrauenswürdig, \dah vom Standpunkt der Sicherheit aus gesehen muss jede Ressource kontinuierlich bewertet werden und darf nur solange genutzt werden, wie sie benötigt wird.
    \item Cloud-Dienste sind ein wesentlicher Bestandteil jedes Unternehmensnetzwerkes geworden und verdeutlichen, dass nicht alle Unternehmensressourcen innerhalb der Unternehmensinfrastruktur liegen.
    \item Alle Verbindungsanfragen von außerhalb des Unternehmens, wie \zb Remote Desktop, müssen \glsdisp{autorisierung}{autorisiert} und \glsdisp{authentizitaet}{authentifiziert} werden.
    Alle Daten müssen mit Respekt, \gls{vertraulichkeit}, \gls{integritaet} und \glsdisp{authentizitaet}{Quellenauthentifizierung} übertragen werden.
    \item Ausgehend der obigen Annahmen ist es essenziell, dass alle Ressourcen und Kommunikation zwischen dem Unternehmen und externer Infrastruktur einer ständigen Sicherheitsstrategie unterliegen muss.
\end{itemize}

\subsection{Vor- und Nachteile von Zero-Trust}\label{subsec:vor-und-nachteile-von-zero-trust}
Eine \ac{zta} bietet verschiedenste Vor- und Nachteile, angefangen von den verbesserten Sicherheitsmetriken, endend bei einer komplexeren Infrastruktur.
Dieser Abschnitt listet und erläutert einzelne dieser Eigenschaften.

\subsubsection{Vorteile}\label{subsubsec:vorteile}
Bei erfolgreicher Implementierung einer \ac{zta} hat jeder Teil des Netzwerkes nur für ein Minimum der Zeit Zugriff auf das Minimum der erforderten Ressourcen.
Dies sorgt dafür, dass ein nahezu umfassender Schutz vor Angriffen besteht, welcher besonders Sicherheitslücken in Systemen, die durch, beabsichtigte oder unbeabsichtigte, Vertrauensbrüche entstehen, angreift\autocite[\vglf][\pagef 146]{Edo-2022}.
Zudem ist eine \ac{zta} flexibler, was die Nutzung von Anwendungen und Geräten betrifft, da die Sicherheitsarchitektur sich nicht auf einzelne Perimeter verlassen muss.\autocites[\vglf][\pagef 28]{shore-2021}[\vglf][]{hunter-2020}
Gleichermaßen stellt eine \ac{zta} eine bessere Einsicht in den Netzverkehr und das Nutzerverhalten dar, was es erneut vereinfacht, Risiken zu erkennen und auf diese zu reagieren.\autocite[\vglf][\pagef 28]{shore-2021}

Besonders Unternehmen, welche ihre Ressourcen primär in Cloud-Systemen verwalten werden einfachere Prozesse in der Umwandlung auf eine \ac{zta} haben, da die Sicherheitsprotokolle bei diesen Systemen meist flexibler einzurichten sind.
Zudem ist eine \ac{zta} in digitalen Unternehmen sehr effektiv, da solche keine klare Perimetergrenze haben, sondern überall existieren wo Kunden, Mitarbeiter oder Partner mit den Diensten interagieren und Daten genutzt werden.
Hierdurch ist eine auf Perimeter basierende Sicherheitsstrategie nicht ausreichend.
Eine \ac{zta} hingegen ermöglicht es, neue Services schnell zu unterstützen, ohne dass eine Verbindung zum gesamten Unternehmesnetzwerk geöffnet wird.
Dies ermöglicht es den Sicherheitsabteilungen an der digitalen Transformation teilzuhaben, anstatt ausschließlich als Verwalter wahrgenommen zu werden.\autocite[\vglf][\pagef 11]{cunningham-2019}

Zusätzlich reduziert eine \ac{zta} die Managementkosten, indem Anzahl und Arten von Sicherheitskontrollen verringert und somit die Anzahl der Managementkonsolen im System reduziert werden.
Dies führt zu einer effizienteren Nutzung von Ressourcen und ermöglicht es den Sicherheitsmitarbeitern in einem Unternehmen, mehr Zeit für substantielle Sicherheitsaktivitäten aufzuwenden.\autocite[\vglf][\pagef 8]{cunningham-2019}

\subsubsection{Nachteile}\label{subsubsec:nachteile}
Während die Vorteile primär auf der technischen Seite einer \ac{zta} liegen, existieren auch Nachteile, welche auf physischer Ebene Auswirkungen zeigen.
So kann sich das Einrichten einer \ac{zta} durch das Erwerben neuer, notwendiger Werkzeuge und Technologien als teuer darstellen.\autocite[\vglf][\pagef 33]{shore-2021}
Zudem ist die Nutzererfahrung mit dem System geringer als gewünscht ausfallen, da jede Anfrage eine neue \gls{autorisierung} und \glsdisp{authentizitaet}{Authentifizierung} erfordert, was besonders zu Anfängen eine nicht vernachlässigbare Zeitdauer in Anspruch nehmen kann.\autocite[\vglf][\pagef 28]{shore-2021}

Darüber hinaus kann die Implementierung einer \ac{zta} komplex und zeitaufwendig sein, sowie signifikante Änderungen in bestehenden Netzwerkinfrastrukturen und Sicherheitsmaßnahmen erfordern.\autocites[\vglf][\pagef 33]{shore-2021}[\vglf][\pagef 11]{buck-2021}
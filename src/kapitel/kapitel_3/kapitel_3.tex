\newpage
\section[Zero-Trust Architektur im Detail]{\ac{zta} im Detail}\label{sec:zta-im-detail}
Da in einer \ac{zta} keinem Gerät und keiner Anwendung vertraut wird, alle aber unterstützt werden, sind verschiedene Maßnahmen notwendig, die erwünschte und oft erforderte Sicherheit zu garantieren.
Dies geschieht durch regelmäßiges Überprüfen der \gls{authentizitaet} und \gls{autorisierung} der Systeme.\autocite[\vglf][\pagef 3]{dsilvia-2021}

\subsection{Komponenten und Architektur von Zero-Trust}\label{subsec:komponenten-und-architektur-von-zero-trust}
Es gibt verschiedene Methoden, eine \ac{zta} einzurichten, von denen viele das Konzept teilen, die Kontrolle nahe an den Anwendungen und Nutzern zu halten, anstatt sie in der Netzwerkinfrastruktur auszulagern.\autocite[\vglf][\pagef 4]{buck-2021}
Drei der Grundkomponenten einer \ac{zta} sind \glsdisp{authentizitaet}{Anwendungsauthentifizierung}, \glsdisp{authentizitaet}{Gerätauthentifizierung} und Vertrauen, da eine \ac{zta} im Gegensatz zu anderen Sicherheitsinfrastrukturen die \gls{authentizitaet} regelmäßig überprüft, die Nutzergeräte überprüft und Widersprüchlichkeiten in Anwendungen der Nutzer überwacht und erkennt.\autocite[\vglf][\pagef 3]{dsilvia-2021}

\subsection{Implementierung von Zero-Trust in Unternehmen}\label{subsec:implementierung-von-zero-trust-in-unternehmen}
Eine \ac{zta} kann sowohl in einem neuen System implementiert werden, als auch in einem bereits bestehenden eingearbeitet werden.
Hierfür nennen D'Silvia und Ambawade\autocite[\vglf][\pagef 3]{dsilvia-2021} folgende Annahmen:
\begin{itemize}
    \item Das LAN innerhalb eines Netzwerkes sollte nicht implizit als vertraute Zone behandelt werden.
    \item Mit dem aktuellen Trend, dass in Unternehmen \ac{byod} eingeführt wird, wird davon ausgegangen, dass Geräte, die mit dem Netzwerk verbunden sind, keine Instanz des Unternehmens sind, da jedes Gerät manipuliert werden kann.
    \item Ressourcen sind niemals vertrauenswürdig, \dah vom Standpunkt der Sicherheit aus gesehen muss jede Ressource kontinuierlich bewertet werden und darf nur solange genutzt werden, wie sie benötigt wird.
    \item Cloud-Dienste sind ein wesentlicher Bestandteil jedes Unternehmensnetzwerkes geworden und verdeutlichen, dass nicht alle Unternehmensressourcen innerhalb der Unternehmensinfrastruktur liegen.
    \item Alle Verbindungsanfragen von außerhalb des Unternehmens, wie \zb Remote Desktop, müssen \glsdisp{autorisierung}{autorisiert} und \glsdisp{authentizitaet}{authentifiziert} werden.
    Alle Daten müssen mit Respekt, \gls{vertraulichkeit}, \gls{integritaet} und \glsdisp{authentizitaet}{Quellenauthentifizierung} übertragen werden
    \item Ausgehend der obigen Annahmen ist es essenziell, dass alle Ressourcen und Kommunikation zwischen dem Unternehmen und externer Infrastruktur einer ständigen Sicherheitsstrategie unterliegen muss.
\end{itemize}

\subsection{Vor- und Nachteile von Zero-Trust}\label{subsec:vor-und-nachteile-von-zero-trust}
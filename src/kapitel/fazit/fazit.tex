\newpage
\section{Fazit}\label{sec:fazit}

Mit dieser Seminararbeit sollten die Auswirkungen einer \ac{zta} auf die Sicherheit eines Systems, sowie auf den Datenverlust dargestellt werden.
Mithilfe einer ausführlichen Analyse der existierenden Literatur wurden diese Auswirkungen veranschaulicht.
Des Weiteren bietet diese Arbeit eine ausführliche Einführung in das Thema \ac{zt} und stellt den Aufbau einer \ac{zta} dar.

Die Analyse der bestehenden Literatur zeigte, dass eine \ac{zta} in verschiedenen Unternehmen weltweit bereits genutzt wird und die Auswirkungen dieser messbar sind.
Besonders im Verwaltungsbereich auf Bundesebene sind \acp{zta} verbreitet, wie in \autoref{subsec:entwicklung-und-historischer-hintergrund} dargestellt wurde.

Die zentralen Auswirkungen einer \ac{zta}, sowie das Potenzial in der weiteren Verbreitung dieser, sind eine Verringerung der Häufigkeit von Cyberangriffen und damit verbundenen Datenverlusten in Unternehmen.
Dies wird durch die strikte Überprüfung von Netzwerkzugriffen bereitgestellt, wodurch eine \ac{zta} eine robuste Sicherheitslösung darstellt.
Dabei muss jedoch beachtet werden, dass eine \ac{zta} einen Einschnitt in die Benutzerfreundlichkeit und Geschwindigkeit des Systems darstellt, da für die Überprüfungen verschiedene Daten eingegeben, gesammelt und gespeichert werden müssen.
Dies stellte im Jahr 2023 für ungefähr ein Drittel der Unternehmen eine Herausforderung beim Aufbau einer \ac{zta} dar, wie in \autoref{fig:zero-trust-challenges} dargestellt ist.

Zusammenfassend lässt sich in Bezug auf die Forschungsfrage \enquote{\myForschungsfrage} sagen, dass eine \ac{zta} eine positive Wirkung auf die Sicherheit in einem System hat, auch wenn Einbußen in Nutzerfreundlichkeit und -produktivität beachtet werden.
Damit diese zusätzlich auf einem gleichen Niveau wie vor der Einführung bleiben, müssen weiterführende Maßnahmen getroffen werden.

\ac{zt} ist ein vergleichsweise altes Konzept, die Grundlagen dafür wurden bereits in den 1990er Jahren gelegt, große Verbreitung erfährt das Konzept jedoch erst seit ein paar Jahren.
Besonders Kindervag, \ac{nist} und \ac{ncsc} trugen stark zur Verbreitung und Forschung an \ac{zt} bei, sodass es heute fast ein fertiges Konzept ist, das Unternehmen nur noch implementieren müssen.\autocites{kindervag-2010}{NIST:800207}{ncsc-2021}

Die vorliegende Seminararbeit hat ausschließlich die direkten Auswirkungen einer \ac{zta}, sowie den grundlegenden Aufbau einer solchen betrachtet.
In einer tiefergehenden Betrachtung kann es daher sinnvoll sein, die Interaktionen einer \ac{zta} in Kombination mit anderen Sicherheitsmaßnahmen zu untersuchen.
Desweiteren kann die Analyse möglicher Maßnahmen zur Steigerung der Nutzerfreundlichkeit und -produktivität in einer \ac{zta} ein Betrachtungsmerkmal einer weiterführenden Arbeit sein.